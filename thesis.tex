%----------------------------------------------------------------
%
%  File    :  thesis.tex
%
%  Authors :  Keith Andrews, IICM, TU Graz, Austria
%             Manuel Koschuch, FH Campus Wien, Austria
%			  Sebastian Ukleja, FH Campus Wien, Austria
% 
%  Created :  22 Feb 96
% 
%  Changed :  14 Oct 2020
%
%  For suggestions and remarks write to: sebastian.ukleja@fh-campuswien.ac.at 
%----------------------------------------------------------------

% --- Setup for the document ------------------------------------

%Class for a book like style:
\documentclass[11pt,a4paper,oneside]{scrbook}
%For a more paper like style use this class instead:
%\documentclass[11pt,a4paper,oneside]{thesis}

%input encoding for windows in utf-8 needed for Ä,Ö,Ü etc..:
\usepackage[utf8]{inputenc}
%input encoding for linux:
%\usepackage[latin1]{inputenc}
%input encoding for mac:
%\usepackage[applemac]{inputenc}

\usepackage[ngerman]{babel}
%for english use this instead:
%\usepackage[english]{babel}

%needed for font encoding
\usepackage[T1]{fontenc}

% want Arial? uncomment next two lines...
%\usepackage{uarial}
%\renewcommand{\familydefault}{\sfdefault}

%some formatting packages
\usepackage[bf,sf]{subfigure}
\renewcommand{\subfigtopskip}{0mm}
\renewcommand{\subfigcapmargin}{0mm}

%For better font resolution in pdf files
\usepackage{lmodern}

\usepackage{url}

%\usepackage{latexsym}

\usepackage{geometry} % define pagesize in more detail

% --- Settings for header and footer ---------------------------------
\usepackage{scrlayer-scrpage}
\clearscrheadfoot
\pagestyle{scrheadings}
\automark{chapter}

%Left header shows chapter and chapter name, will not display on first chapter page use \ihead*{\leftmark} to show on every page
\ihead{\leftmark} 	
%\ohead*{\rightmark}	%optional right header
\ifoot*{Markus Mayer}	%left footer shows student name
\ofoot*{\thepage}		%right footer shows pagination
%---------------------------------------------------------------------

\usepackage{colortbl} % define colored backgrounds for tables

\usepackage{courier} %for listings
\usepackage{listings} % nicer code formatting
\lstset{basicstyle=\ttfamily,breaklines=true}

\usepackage{graphicx}
  \pdfcompresslevel=9
  \pdfpageheight=297mm
  \pdfpagewidth=210mm
  \usepackage[         % hyperref should be last package loaded
    pdftex, 		   % needed for pdf compiling, DO NOT compile with LaTeX
    bookmarks,
    bookmarksnumbered,
    linktocpage,
    pagebackref,
    pdfview={Fit},
    pdfstartview={Fit},
    pdfpagemode=UseOutlines,                 % open bookmarks in Acrobat
  ]{hyperref}
\DeclareGraphicsExtensions{.pdf,.jpg,.png}
\usepackage{bookmark}

\usepackage[title]{appendix}

%paper format
\geometry{a4paper,left=30mm,right=25mm, top=30mm, bottom=30mm}

\setlength{\parskip}{3pt plus 1pt minus 0pt}       % vert. space before a paragraph

\setcounter{tocdepth}{1}        % lowest section level entered in ToC
\setcounter{secnumdepth}{2}     % lowest section level still numbered

%Start of your document beginning with title page
\begin{document}

% --- Main Title Page ------------------------------------------------
\begin{titlepage}
\frontmatter
\begin{picture}(50,50)
\put(-70,40){\hbox{\includegraphics{images/logo.png}}}
\end{picture}

\vspace*{-5.8cm}

\begin{center}

\vspace{6.2cm}

\hspace*{-1.0cm} {\LARGE \textbf{Wird Kotlin in Zukunft Java in der Android App Entwicklung ablösen? \\}}
\vspace{0.2cm}
\hspace*{-1.0cm} Untertitel \\

\vspace{2.0cm}

\hspace*{-1.0cm} { \textbf{Bachelorarbeit\\}}

\vspace{0.65cm}

\hspace*{-1.0cm} Eingereicht in teilweiser Erfüllung der Anforderungen zur Erlangung des akademischen Grades: \\

\vspace{0.65cm}

\hspace*{-1.0cm} \textbf{Bachelor of Science in Engineering\\}

\vspace{0.65cm}

\hspace*{-1.0cm} an der FH Campus Wien \\
\vspace{0.2cm}
\hspace*{-1.0cm} Studienfach: Computer Science and Digital Communications \\

\vspace{1.6cm}

\hspace*{-1.0cm} \textbf{Autor:} \\
\vspace{0.2cm}
\hspace*{-1.0cm} Markus Mayer \\

\vspace{0.7cm}

\hspace*{-1.0cm} \textbf{Matrikelnummer:}\\
\vspace{0.2cm}
\hspace*{-1.0cm} 52006537 \\

\vspace{0.7cm}

\hspace*{-1.0cm} \textbf{Betreuer:} \\
\vspace{0.2cm}
\hspace*{-1.0cm} MSc René Goldschmid \\

\vspace{0.7cm}

% Reviewer if needed:
%\hspace*{-1.0cm} \textbf{Reviewer: (optional)} \\
%\vspace{0.2cm}
%\hspace*{-1.0cm} Titel Vorname Nachname \\


\vspace{1.0cm}

\hspace*{-1.0cm} \textbf{Datum:} \\
\vspace{0.2cm}
\hspace*{-1.0cm} 25.09.2022 \\

\end{center}
\end{titlepage}

\newpage

\setcounter{page}{1}

\vspace*{16cm}

% --- Declaration of authorship --------------------------------------------
\hspace*{-0.7cm} \underline{Erklärung der Urheberschaft:}\\\\
Ich erkläre hiermit diese Bachelorarbeit eigenständig verfasst zu haben. Ich habe keine anderen Quellen, als die in der Arbeit gelisteten verwendet, noch habe ich jegliche unerlaubte Hilfe in Anspruch genommen\\\\
Ich versichere diese Bachelorarbeit in keinerlei Form jemandem Anderen oder einer anderen Institution zur Verfügung gestellt zu haben, weder in Österreich noch im Ausland.\\\\
Weiters versichere ich, dass jegliche Kopie (gedruckt oder digital) identisch ist.
\\\\\\
Datum: \hspace{6cm} Unterschrift:\\

% --- English Abstract ----------------------------------------------------
\cleardoublepage
\chapter*{Abstract}
(E.g. ``This thesis investigates...'')


% --- German Abstract ----------------------------------------------------

\cleardoublepage
\chapter*{Kurzfassung}
(Z.B. ``Diese Arbeit untersucht...'')

% --- Abbrevations ----------------------------------------------------
\newpage\noindent
\chapter*{Abkürzungen}
\vspace{0.65cm}

\begin{table*}[htbp]
		\begin{tabular}{ll}
			
		\end{tabular}
\end{table*}

% --- Key terms ----------------------------------------------------
\newpage
\chapter*{Schlüsselbegriffe}
\vspace{0.65cm}

\begin{itemize}
	\setlength{\itemsep}{0pt}
	\item[] Android
	\item[] Java
	\item[] Kotlin
\end{itemize}

% --- Table of contents autogenerated ------------------------------------
\newpage
\tableofcontents

% --- Begin of Thesis ----------------------------------------------------
\mainmatter
\chapter{Einführung}
\label{chap:Einführung}

\section{Motivation}
\label{sec:Motivation}
Warum ist diese Arbeit interessant?

\section{Relevanz}
\label{sec:Relevanz}
Inwiefern ist diese Arbeit neu und wichtig?

\chapter{Hauptteil}
\label{chap:Hauptteil}

\section{Android}
\label{sec:Android}
Generelle Übersicht zu Android

\section{Java}
\label{sec:Java}
Generelle Übersicht zu Java

\section{Kotlin}
\label{sec:Kotlin}
Generelle Übersicht zu Kotlin

\section{Allgemeine Unterschiede}
\label{sec:AllgemeineUnterschiede}
Allgemeine Unterschiede zwischen Java und Kotlin wie zum Beispiel Semicolons, Null Safety,  Type Inference, Extension Functions, ...


\section{Entwicklungseffizienz}
\label{sec:Entwicklungseffizienz}
Analyse ob Kotlin eine bessere Entwicklungseffizienz hat. 

\section{Wartbarkeit}
\label{sec:Wartbarkeit}
Vergleich der Wartbarkeit der beiden Sprachen.

\section{Leichtigkeit der Entwicklung}
\label{sec:LeichtigkeitDerEntwicklung}
Betrachtung der Leichtigkeit der Sprachen bei der Entwicklung.

\section{Laufzeit Effizienz}
\label{sec:LaufzeitEffizienz}
Vergleich der Laufzeit Effizienz der beiden Sprachen.

\section{Verwandte Arbeiten}
\label{sec:VerwandteArbeiten}


\chapter{Schluss}
\label{chap:Schluss}

\section{Zusammenfassung}
\label{sec:Zusammenfassung}
Wichtigsten Aussagen der Arbeit wiederholen und miteinander in Beziehung bringen

\section{Bewertung und Ausblick}
\label{sec:BewertungUndAusblick}
Bewertung der verschiedenen Faktoren und Erkenntnisse
Ideen für weitere Arbeiten

\newpage

% --- Bibliography ------------------------------------------------------

%IEEE Citation [1]:
\bibliographystyle{IEEEtran}
%for alphanumeric citation eg.: [ABC19]
%\bibliographystyle{alpha}

% List references I definitely want in the bibliography,
% regardless of whether or not I cite them in the thesis.

\newpage
\addcontentsline{toc}{chapter}{Bibliographie}
\bibliography{testBib}

\newpage

% --- List of Figures ----------------------------------------------------

\addcontentsline{toc}{chapter}{Abbildungen}
\listoffigures


% --- List of Tables -----------------------------------------------------

\newpage
\addcontentsline{toc}{chapter}{Tabellen}
\listoftables

% --- Appendix A -----------------------------------------------------

\newpage
\backmatter
\appendix
\begin{appendices}
\chapter{Appendix}

(Hier können Schaltpläne, Programme usw. eingefügt werden.)

\clearpage
\end{appendices}

\end{document}
